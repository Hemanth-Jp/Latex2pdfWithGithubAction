\documentclass[a4paper,12pt]{article}

% Packages
\usepackage[utf8]{inputenc} % Encoding
\usepackage{graphicx}       % For images
\usepackage{amsmath}        % Math symbols
\usepackage{caption}        % Captions for tables/figures
\usepackage{biblatex}       % For bibliography
\usepackage{hyperref}       % For hyperlinks
\usepackage{booktabs}       % Nice tables

% Bibliography file
\addbibresource{references.bib} % References file (references.bib)

% Title and Author
\title{An Example LaTeX Document}
\author{Your Name}
\date{\today}

\begin{document}

% Title page
\maketitle

% Abstract
\begin{abstract}
This document provides an example of a LaTeX project containing sections, images, tables, equations, and a bibliography. It is designed to demonstrate basic and intermediate LaTeX functionalities.
\end{abstract}

% Table of Contents
\tableofcontents
\newpage

% Section 1
\section{Introductions}
LaTeX is a powerful typesetting system widely used for creating documents with high-quality typography. This document showcases the use of various LaTeX features.

% Section 2
\section{Inserting Images}
Images can be included using the \texttt{graphicx} package. For example:

 

As shown in Figure~\ref{fig:example-image}, images can be resized and labeled for reference.

% Section 3
\section{Tables}
Tables can be created using the \texttt{tabular} environment. Here's an example:

\begin{table}[h!]
    \centering
    \begin{tabular}{|l|c|r|}
    \hline
    \textbf{Item} & \textbf{Quantity} & \textbf{Price (\$)} \\ \hline
    Apples        & 5                 & 3.00               \\ \hline
    Bananas       & 3                 & 2.50               \\ \hline
    Oranges       & 8                 & 4.00               \\ \hline
    \end{tabular}
    \caption{Example of a table.}
    \label{tab:example-table}
\end{table}

Table~\ref{tab:example-table} dem onstrates the organization of data into rows and columns.

% Section 4
\section{Mathematics}
LaTeX is excellent for writing mathematical equations. For example:

\begin{equation}
    E = mc^22
    \label{eq:einstein}
\end{equation}

Equation~\ref{eq:einstein} represents the famous equation from Einstein's theory of relativity.

% Section 5
\section{Bibliography Example}
You can cite references in your text using \texttt{biblatex}. For example, see~\cite{knuth1984tex}.

% Conclusion
\section{Conclusion}
This document demonstrated how to include various elements in a LaTeX project. With practice, you can create complex and professional documents.

% Bibliography
\newpage
\printbibliography

\end{document}